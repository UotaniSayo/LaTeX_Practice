\documentclass[12pt,a4paper]{article}

\usepackage{sslab_ctrl_rinko}
\usepackage[dvips]{graphicx}
\usepackage{latexsym}
\usepackage{enumerate}
\usepackage{plext}
\usepackage{float}
\usepackage{amsmath,amssymb}
\usepackage{amsthm}
\usepackage{ascmac}
\usepackage{mathrsfs}
\usepackage{bm}


% \usepackage{cite}

\allowdisplaybreaks

\title{title}

\begin{document}
\leftline{名前:韓心又}
\leftline{学生番号:1751204}

\section{本日のスライドの空白を埋め、理由も付記せよ。}
問1
$\\\because\ddot{y} =u, y(0)=0, \dot{y} (0)=0\\
\therefore y=0.5t^2$\\
\\$(s\bm{I}-\bm{A})^{-1}=
\begin{bmatrix}
s^{-1} & s^{-2}\\
0 & s^{-1}
\end{bmatrix}\\$
$\because\mathscr{L} ^{-1}[s^{-1}]=1,\mathscr{L} ^{-1}[s^{-2}]=t\\$
$\therefore e^{\bm{A}t}=
\begin{bmatrix}
1 & t\\
0 & 1
\end{bmatrix}
\Rightarrow \bm{A}_\Delta=e^{\bm{A}\Delta}=
\begin{bmatrix}
1 & \Delta\\
0 & 1
\end{bmatrix}
=
\begin{bmatrix}
1 & 0.1\\
0 & 1
\end{bmatrix}\\$
$\bm{b}_\Delta=\int_0^\Delta e^{\bm{A}\tau}\bm{b}\intd\tau =
\begin{bmatrix}
0.5\Delta^2\\
\Delta
\end{bmatrix}\\$
\\
\\$y[k]=
\begin{bmatrix}
1 & 0
\end{bmatrix}
\sum\nolimits_{\tau=0}^{k-1} 
\begin{bmatrix}
0.5+(k-1-\tau)\Delta^2\\
\Delta
\end{bmatrix}\\
=\cdots\\
=0.5k^2\Delta^2=0.005k^2$
\\
\\
問2 (2)\\
固有ベクトルは
$\\\bm{v}_1=
\begin{bmatrix}
1\\
-1
\end{bmatrix},
\bm{v}_2=
\begin{bmatrix}
1\\
1
\end{bmatrix}$\\
$\therefore$対角変換行列は
$\\\bm{T}=
\begin{bmatrix}
1 & 1\\
-1 & 1
\end{bmatrix}$
\\
$\therefore\bm{A}_\Delta^k=\left(\bm{T}
\begin{bmatrix}
0 & 0\\
0 & 2
\end{bmatrix}
\bm{T}^{-1}\right)^k=2^{k-1}
\begin{bmatrix}
1 & 1\\
1 & 1
\end{bmatrix}$

$\\\bm{x}[k]=2^{k-1}
\begin{bmatrix}
1 & 1\\
1 & 1
\end{bmatrix}
\begin{bmatrix}
\alpha\\
\beta
\end{bmatrix}=2^{k-1}
\begin{bmatrix}
\alpha+\beta\\
\alpha+\beta
\end{bmatrix}$
\\
\\
\\
問3 (1)
$\\e^{\bm{A}t}=
\begin{bmatrix}
1 & 1\\
3j & -3j
\end{bmatrix}
\begin{bmatrix}
e^{3jt} & 0\\
0 & e^{-3jt}
\end{bmatrix}
\begin{bmatrix}
1 & 1\\
3j & -3j
\end{bmatrix}^{-1}$
$\\=
\begin{bmatrix}
1 & 1\\
3j & -3j
\end{bmatrix}
\begin{bmatrix}
\mathrm{cos}3t+j\mathrm{sin}3t & 0\\
0 & \mathrm{cos}3t-j\mathrm{sin}3t
\end{bmatrix}
\begin{bmatrix}
\frac{1}{2} & -\frac{1}{6}j\\
\frac{1}{2} & \frac{1}{6}j
\end{bmatrix}$
$\\=
\begin{bmatrix}
\mathrm{cos}3t & \frac{1}{3}\mathrm{sin}3t\\
-3\mathrm{sin}3t & \mathrm{cos}3t
\end{bmatrix}$
$\\\because \bm{x}(0)=
\begin{bmatrix}
1\\
0
\end{bmatrix}$
$\\\therefore e^{\bm{A}t}\bm{x}(0)=
\begin{bmatrix}
\mathrm{cos}3t\\
-3\mathrm{sin}3t
\end{bmatrix}$
\\
\\
(2)
$\\\because \lambda^2+k_2\lambda+9+k_1=\lambda^2+2\lambda+5$
$\\\therefore k_1=-4, k=2$
\\
\\
\\
スライド13
$\\\Omega=\frac{100}{501}
\begin{bmatrix}
\frac{401}{501} & -\frac{200}{501}\\
-\frac{200}{501} & \frac{101}{501}
\end{bmatrix}$
\\
\\
\\
スライド14
$\begin{bmatrix}
1.01 & 2\\
2 & 4.01
\end{bmatrix}
\frac{100}{501}
\begin{bmatrix}
\frac{401}{501} & -\frac{200}{501}\\
-\frac{200}{501} & \frac{101}{501}
\end{bmatrix}$
$\\=\frac{1}{501}
\begin{bmatrix}
101 & 200\\
200 & 401
\end{bmatrix}
\begin{bmatrix}
\frac{401}{501} & -\frac{200}{501}\\
-\frac{200}{501} & \frac{101}{501}
\end{bmatrix}$
\\
\\
\\
スライド15
$\\\left(\bm{\Omega}+\bm{\xi\xi}^T\right)^{-1}=\cdots=\frac{1}{15}
\begin{bmatrix}
14 & -2 & -3\\
-2 & 11 & -6\\
-3 & -6 & 6\\
\end{bmatrix}$

$\\\frac{\bm{\Omega}^{-1}\bm{\xi\xi}^T\bm{\Omega}^{-1}}{1+\bm{\xi}^T\bm{\Omega}^{-1}\bm{\xi}}=\frac{1}{15}
\begin{bmatrix}
1 & 2 & 3\\
2 & 4 & 6\\
3 & 6 & 9
\end{bmatrix}$

$\\
\begin{bmatrix}
2 & 2 & 3\\
2 & 5 & 6\\
3 & 6 & 10
\end{bmatrix}
\frac{1}{15}
\begin{bmatrix}
14 & -2 & -3\\
-2 & 11 & -6\\
-3 & -6 & 6\\
\end{bmatrix}=\bm{I}_3$
\\
\\
\\
スライド19
$\\\bm{\dot{x}}=
\begin{bmatrix}
-\frac{R}{L} & 0 & -K_e\\
0 & 0 & 1\\
\frac{K_\tau}{JL} & 0 & -\frac{D}{J}
\end{bmatrix}\bm{x}+
\begin{bmatrix}
1\\
0\\
0\\
\end{bmatrix}u$
$K_e=0$にすれば、行列$\bm{A}$は
$\\
\begin{bmatrix}
-\frac{R}{L} & 0 & 0\\
0 & 0 & 1\\
\frac{K_\tau}{JL} & 0 & -\frac{D}{J}
\end{bmatrix}$
$\\\therefore \left(s\bm{I}-\bm{A}\right)=
\begin{bmatrix}
s+\frac{R}{L} & 0 & 0\\
0 & s & -1\\
-\frac{K_\tau}{JL} & 0 & s+\frac{D}{J}
\end{bmatrix}$
\\
方程式
$\\\left(s\bm{I}-\bm{A}\right)
\begin{bmatrix}
\alpha(s)\\
\beta(s)\\
\gamma(s)
\end{bmatrix}=
\begin{bmatrix}
1\\
0\\
0
\end{bmatrix}$
\\
から
$\\\alpha(s)=\frac{L}{Ls+R}
\\\beta(s)=\frac{K_\tau}{LJs^3+(LD+JR)s^2+RDs}
\\\gamma(s)=\frac{K_\tau}{LJs^2+(LD+JR)s+RD}$
\\
がわかる
$\\\therefore G(s)=(0\quad 1\quad 0)
\begin{bmatrix}
\alpha(s)\\
\beta(s)\\
\gamma(s)
\end{bmatrix}=\frac{K_\tau}{LJs^3+(LD+JR)s^2+RDs}$
\newpage
\section{システム分析}
$\bm{\dot{x}}=\bm{Ax}+\bm{b}u,$
$\\y=\bm{cx}$
\\
$\\A=
\begin{bmatrix}
0 & 1\\
-4 & 0
\end{bmatrix},b=
\begin{bmatrix}
0\\
1
\end{bmatrix},c=
\begin{bmatrix}
1 & 0
\end{bmatrix}$
$\\e^{\bm{A}t}=\mathscr{L}^{-1}\left[\left(s\bm{I}-\bm{A}\right)^{-1}\right]$
$\\\left(s\bm{I}-\bm{A}\right)^{-1}=\frac{1}{s^2+4}
\begin{bmatrix}
s & 1\\
-4 & s
\end{bmatrix}$
$\\\therefore e^{\bm{A}t}=
\begin{bmatrix}
\mathrm{cos}2t & 0.5\mathrm{sin}2t\\
2\mathrm{sin}2t & \mathrm{cos}2t
\end{bmatrix}$
\\
\\
$\Delta=0.1$、離散システムにすれば
$\\\bm{A}_\Delta=
\begin{bmatrix}
\mathrm{cos}0.2 & 0.5\mathrm{sin}0.2\\
2\mathrm{sin}0.2 & \mathrm{cos}0.2
\end{bmatrix}$
$\\\bm{b}_\Delta=
\begin{bmatrix}
0.25-0.25\mathrm{cos}0.2\\
0.5\mathrm{sin}0.2
\end{bmatrix}$
\\
\\
$\bm{A}$の固有値について
$\\|\lambda\bm{I}-\bm{A}|=
\begin{vmatrix}
\lambda-\mathrm{cos}0.2 & -0.5\mathrm{sin}0.2\\
2\mathrm{sin}0.2 & \lambda-\mathrm{cos}0.2
\end{vmatrix}=\left(\lambda-\mathrm{cos}0.2\right)^2-\mathrm{sin}^20.2=0$
\\
$\Rightarrow \lambda^2-2\lambda\mathrm{cos}0.2+1=0$
$\\\lambda=\mathrm{cos}0.2\pm j\mathrm{sin}0.2$
$\\\because |\lambda|=1$
$\\\therefore$このシステムは漸近安定でありません。


\end{document}
